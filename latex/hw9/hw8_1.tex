% This file is a valid PHP file and also a valid LaTeX file
% When processed with LaTeX, it will generate a blank template
% Loading with PHP will fill it with details

\documentclass[12pt,a4paper]{article}
% Required for proper escaping
\usepackage[left=2.5cm,right=2.5cm,top=2.5cm,bottom=2.5cm]{geometry}
%\usepackage{ctex}
%\XeTeXlinebreaklocale "zh"
%\XeTeXlinebreakskip = 0pt plus 1pt
\usepackage{CJK}
\usepackage{graphicx}

% Make placeholders visible
\newcommand{\placeholder}[1]{\textbf{$<$ #1 $>$}}

% Defaults for each variable
\newcommand{\idnumber}{\placeholder{Student ID}}
\newcommand{\moodleid}{\placeholder{ID}}


% Fill in
% <?php echo "\n" . "\\renewcommand{\\idnumber}{" . CLatexTemplate::escape($data['idnumber']) . "}\n"; ?>
% <?php echo "\n" . "\\renewcommand{\\moodleid}{" . CLatexTemplate::escape($data['moodleid']) . "}\n"; ?>

% LaTeX code for the invoice
\usepackage{tabularx}
\setlength{\parindent}{0pt}
\pagestyle{empty}
\usepackage{amsmath}
\usepackage{amssymb}


\begin{document}
\begin{CJK}{UTF8}{bsmi}
\begin{flushleft}Introduction to Scientific Computing Software HW8
\\Student ID : \idnumber{}\end{flushleft}

牛頓法求根(二)
\begin{enumerate}
\item 請使用上次作業寫好的 $f(x)=x^{-2}\tan x$ 函數及其微分 myfun.m 及 myfund.m,完成本次作業。
\item 請利用上次的 mynewton.m 函數進修改,參數:起始值、誤差($Tolx$)、最大迭代次數,回傳值:計算結果、函數值($fval$)。
\item 當 $x_n$ 和新迭代點 $x_{n+1}$ 距離小於 $Tolx$ 時,判定牛頓法收斂,回傳。
\begin{itemize}
\item 函數值($fval$)過大($|fval|>1e-5$),\\以 warning 顯示警告:\texttt{Converge,but fval larger than 1e-5}
\end{itemize}
\item 當迭代次數超過最大迭代次數時,判定迭代失敗。
\begin{itemize}
\item 以error函數顯示錯誤訊息:\texttt{Failed to converge}
\end{itemize}
\item 請寫一個 hw9.m,執行 mynewton 函數,起始點為 $x_0=4$。

\end{enumerate}
\end{CJK}
\end{document}
