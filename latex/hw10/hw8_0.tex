% This file is a valid PHP file and also a valid LaTeX file
% When processed with LaTeX, it will generate a blank template
% Loading with PHP will fill it with details

\documentclass[12pt,a4paper]{article}
% Required for proper escaping
\usepackage[left=2.5cm,right=2.5cm,top=2.5cm,bottom=2.5cm]{geometry}
%\usepackage{ctex}
%\XeTeXlinebreaklocale "zh"
%\XeTeXlinebreakskip = 0pt plus 1pt
\usepackage{CJK}
\usepackage{graphicx}
\usepackage{listings}
\usepackage{color}
\definecolor{mylilas}{RGB}{170,55,241}


\lstset{language=Matlab,%
    %basicstyle=\color{red},
    basicstyle=\ttfamily,
    breaklines=true,%
    morekeywords={matlab2tikz},
    keywordstyle=\color{blue},%
    morekeywords=[2]{1}, keywordstyle=[2]{\color{black}},
    identifierstyle=\color{black},%
    stringstyle=\color{mylilas},
    commentstyle=\color{mygreen},%
    showstringspaces=false,%without this there will be a symbol in the places where there is a space
    numbers=left,%
    numberstyle={\tiny \color{black}},% size of the numbers
    numbersep=9pt, % this defines how far the numbers are from the text
    emph=[1]{for,end,break},emphstyle=[1]\color{red}, %some words to emphasise
    %emph=[2]{word1,word2}, emphstyle=[2]{style},
}
% Make placeholders visible
\newcommand{\placeholder}[1]{\textbf{$<$ #1 $>$}}

% Defaults for each variable
\newcommand{\idnumber}{\placeholder{Student ID}}
\newcommand{\moodleid}{\placeholder{ID}}


% Fill in
% <?php echo "\n" . "\\renewcommand{\\idnumber}{" . CLatexTemplate::escape($data['idnumber']) . "}\n"; ?>
% <?php echo "\n" . "\\renewcommand{\\moodleid}{" . CLatexTemplate::escape($data['moodleid']) . "}\n"; ?>

% LaTeX code for the invoice
\usepackage{tabularx}
\setlength{\parindent}{0pt}
\pagestyle{empty}
\usepackage{amsmath}
\usepackage{amssymb}


\begin{document}
\begin{CJK}{UTF8}{bsmi}
\begin{flushleft}Introduction to Scientific Computing Software HW10
\\Student ID : \idnumber{}\end{flushleft}

本次作業要開始使用 Python 3 囉!這次作業希望大家使用 Spyder 來練習寫 Python,作業請繳交 \texttt{.py} 檔
\begin{itemize}
\item 請將以下 MATLAB 畫圖程式翻譯成 Python 版本
\begin{enumerate}
\item 無 function 版本
\begin{lstlisting}[frame=single,caption=prob1.m]
fs=44100; 
x=zeros(fs+1,1);
y=zeros(fs+1,1);
xa=0;
xb=10;
dx=(xb-xa)/fs;

for ii=1:fs+1
    x(ii)=xa+dx*ii;
    y(ii)=2*cos(pi*x(ii))+cos(2*pi*x(ii));
end

plot(x,y)
\end{lstlisting}
\item 有 function 版本
\begin{lstlisting}[frame=single,caption=myfun.m]
function y=myfun(x)
    y=2*cos(pi*x)+cos(2*pi*x);
end
\end{lstlisting}
\begin{lstlisting}[frame=single,caption=prob2.m]
fs=44100; 
x=zeros(fs+1,1);
y=zeros(fs+1,1);
xa=0;
xb=10;
dx=(xb-xa)/fs;

for ii=1:fs+1
    x(ii)=xa+dx*ii;
    y(ii)=myfun(x(ii));
end

plot(x,y)
\end{lstlisting}

\end{enumerate}
\newpage
\item Hint:你會需要 import 的套件,和一些關鍵的部分。(Python是靠縮排來辨認程式的區塊!沒有分號、大括號、end等東西把程式包起來)
\begin{lstlisting}[frame=single,caption=prob1.py,language=Python]
import matplotlib.pyplot as plt
import math

x=[0]*10

math.cos(2*math.pi)

def myfun(t):
    y=t**2
    return y
    
for i in range(0,10):
    print(myfun(i))
\end{lstlisting}
\end{itemize}
\end{CJK}
\end{document}
