% This file is a valid PHP file and also a valid LaTeX file
% When processed with LaTeX, it will generate a blank template
% Loading with PHP will fill it with details

\documentclass[12pt,a4paper]{article}
% Required for proper escaping
\usepackage[left=2.5cm,right=2.5cm,top=2.5cm,bottom=2.5cm]{geometry}

\usepackage{CJKutf8}
\usepackage{CJK}

% Make placeholders visible
\newcommand{\placeholder}[1]{\textbf{$<$ #1 $>$}}

% Defaults for each variable
\newcommand{\idnumber}{\placeholder{Student ID}}
\newcommand{\moodleid}{\placeholder{ID}}


% Fill in
% <?php echo "\n" . "\\renewcommand{\\idnumber}{" . CLatexTemplate::escape($data['idnumber']) . "}\n"; ?>
% <?php echo "\n" . "\\renewcommand{\\moodleid}{" . CLatexTemplate::escape($data['moodleid']) . "}\n"; ?>

% LaTeX code for the invoice
\usepackage{tabularx}
\setlength{\parindent}{0pt}
\pagestyle{empty}
\usepackage{amsmath}
\usepackage{amssymb}


\begin{document}
\begin{CJK}{UTF8}{bsmi}
\begin{flushleft}科學計算軟體 HW3
\\Student ID : \idnumber{}\end{flushleft}

請使用 MATLAB 完成以下題目 : 
\begin{enumerate}
\item 拍音(Beating)\\
$\circledcirc$以下圖形的 $x$ 範圍皆為$[0,2\pi]$。\\
$\circledcirc$圖片標題:100Hz+110Hz $\Rightarrow f_{beat}=5$Hz\\
$\circledcirc$在右下角放置 Legend,標示兩小題的圖形。
\begin{enumerate} 
\item 畫出 $\sin(100x)$ 和 $\sin(110x)$ 疊合後的圖形。

\item 用「紅色」畫出 $2\cos(5x)$ 和 $-2\cos(5x)$ 的圖形,並疊在(a)小題的圖上。
\end{enumerate}
%\item 泰勒多項式\\$\circledcirc$以下圖形的 $x$ 範圍皆為$[-5,5]$。\\
%$\circledcirc$圖片標題:Tyler polynimial of $e^x$ at 0。\\
%$\circledcirc$在左上角放置 Legend,標示各階泰勒多項式及原函數。
%\begin{enumerate} 
%\item 畫出 $e^x$ 的圖形。
%\item 畫出 $e^x$ 在 $x=0$ 的3階($T_3(x)$)、4階($T_4(x)$)、5階($T_5(x)$)泰勒多項式,並疊在(a)小題的圖上。
%\end{enumerate}
\end{enumerate}
\end{CJK}
\end{document}
